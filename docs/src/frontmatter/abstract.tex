% ======================================
% How to write an abstract?
% ======================================
% Ref: https://www.anu.edu.au/students/academic-skills/research-writing/journal-article-writing/writing-an-abstract

% Whereas the purpose of an introduction is to broadly introduce your topic and your key message, the purpose of an abstract is to give an overview of your entire project, in particular its findings and contribution to the field. An abstract should be a standalone summary of your paper, which readers can use to decide whether it's relevant to them before they dive in to read the paper.

% Usually an abstract includes the following.
%     A brief introduction to the topic that you're investigating.
%     Explanation of why the topic is important in your field/s.
%     Statement about what the gap is in the research.
%     Your research question/s / aim/s.
%     An indication of your research methods and approach.
%     Your key message.
%     A summary of your key findings.
%     An explanation of why your findings and key message contribute to the field/s.

% It should answer these questions:
% What is your paper about?
% Why is it important?
% How did you do it?
% What did you find?
% Why are your findings important?

% Other version have this simplified structure:
% Background, Methodology, Results, Conclusion

% =================================
% ABSTRACT START
% =================================

\abstract
Formal specifications are a useful method of communicating system descriptions due to their lack of ambiguity,
but they tend to take time to write and require skills few people possess.
In this paper, we propose a way to derive formal specifications from commonly generated informal graphical models.
With our tool, the UML to C2KA Converter (U2C),
we reduce the barriers of formal methods to engineer quality systems.
We managed to read UML State Diagrams to automatically produce Communicating Concurrent Kleene Algebra (C2KA) specifications.
These specifications can be fed to C2KA model checkers, like the Implicit Interactions Analysis Tool (IIAT).
By improving the accessibility of these model checkers, we can identify vulnerabilities and faults earlier in the design process.
This should reduce design costs and reduce operational damages by helping engineers build safer and more secure systems.
\\
\textbf{Keywords}: C2KA, UML, Finite State Machine, state diagram, formal methods, model checking, model driven engineering.
% =================================
% ABSTRACT END
% =================================
