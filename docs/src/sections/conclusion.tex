We wanted to find a way to produce C2KA system specifications more easily.
C2KA specifications are a useful formalism to analyze systems, but they take time and knowledge to produce.
As such, other methods to gain assurance in a system are used.
These are typically lower in cost but cannot provide the same degree of assurance (like unit testing code).

We believed that we could leverage a more universal skill than C2KA specification writing:
Creating visual diagram representations of the system.
Visual diagrams have the added benefit of being artefacts that are useful on their own already.
This means that users of our tool may already be using them or benefitting from their creation.

The prototype we've built takes in UML state diagrams and can produce C2KA specification files for each agent in the system.
On their own, they can provide a comprehensive overview of the possible behaviors of an agent in the system.
When fed to other analysis tools like the IIAT,
these specifications can also be used to analyze the system for implicit interactions.
These implicit interactions may be the source of hidden vulnerabilities in the system,
which we can now help detect easily.

\subsection{Further Recommendations}\label{subsec:further-recommendations}
There are a few gaps in our prototype compared to our requirements.
For an up-to-date detailed account, our GitHub issues should be the best source in case of any changes after this report was written.
For consistency, we will follow the issue categories we used in GitHub to discuss our recommendations.
See section~\ref{subsubsec:proj-mngmnt} for definitions of these categories.

\subsubsection{Verification \& Validation}\label{subsubsec:rec-v-v}
We are lacking a formal definition of coverage criteria.
The test strategy we defined does a good job of capturing the idea behind how we try to cover tests.
Formalizing coverage criteria would be taking these ideas and formulating them in a way that we can systematically
test our code coverage and build a report to see if we conform to our desired coverage.
This report would also be an important baseline for assurance in our software.
Checking coverage is done manually at the moment, we create GitHub issues if we notice missing coverage (easy to miss).

We also lack system test (validation tests) variety.
It would have been nice to have system tests to test other metrics than the ``Accuracy'' of our system.
We have no stress tests for performance, for example.
This is partially due to a late definition of our non-functional requirements,
as explained later in section~\ref{subsubsec:req-elicit-refl}.

In general, more tests are needed.
We are currently lacking coverage (verification).
More system tests increase assurance in our system (validation); we have very few of them at the moment.

\subsubsection{Enhancements}
The main feature we lack is collaboration diagram processing for IIAT parameter extraction.
We cannot produce an intended sequence of inputs for the IIAT, which was part of our initial set of functional requirements.

Another notable mention was performance improvement-related fixes.
Our testing did not yet support proper analysis for these changes; thus, work on them did not get started either.

We entertained the idea of having a separate model checker module as part of our tool.
We could have a separate pipeline that could be called independently to check the validity of the inputs.
We would define rules in the module to check the assumptions we defined in section~\ref{subsubsec:input-specification}.
This would help prevent potential unexpected behaviors (false positive outputs),
and it could help improve the error messages we give to users to make fixing user diagrams easier.
The side benefit is that by trying to make the artefacts work for our tool,
we may improve diagram quality by automatically reviewing them through our model checker.

\subsubsection{Bugs}
There are none (that we know of!).

In this context, we will count bugs as faults that could lead to an incorrect output being produced.
If the program crashes before producing the output, we may count it as a missing feature, depending on the cause.
A well-formed input that crashes means we have not developed the tools to process it yet.
Collaboration diagrams could fall in this category, although we have not yet defined what is a well-formed collaboration diagram either.
Otherwise, it is good if the program crashes because it means that it detected a fault in the input.
If it did not, it would be a bug because the output is faulty (false positive output).

Our philosophy as we were finishing our prototype was to focus on a minimum viable product with well-tested essential features.
We did not want to add many unessential features.
We knew we would not have the proper time to identify faults if we increased our scope.
This could have led to unsatisfied critical non-functional requirements, like accuracy.
The goal is that we may not be able to output everything, but what we do produce can be relied on (no false positive).

\subsection{Reflections}\label{subsec:reflections}
\subsubsection{Requirement Elicitation Phase}\label{subsubsec:req-elicit-refl}
This section refers to how we collected background information and determined our requirements from there.
This does not refer to how we managed requirements after we started the implementation.
The issue with our approach is two-fold.
We took too much time (we did not start implementation work until after the progress report), and it was not productive given the time spent on it.
Although the User Requirement list in section~\ref{subsubsec:user-reqs} is adequate, it was never formally defined until the report was written.

At multiple times during implementation, progress halted due to a lack of C2KA knowledge,
or we had to rewrite code from a poor initial understanding.
We also did not approve or trace any requirements other than our functional objectives.
These functional objectives were based on a specific pipeline configuration, which we decided to change later,
and thus were too low-level to base our functional requirements on them.
The non-functional objectives of our system became even more implicit and were subject to change since there was no traceability for them.
This means any decisions regarding trade-offs in the program were decided based on what felt best at the time,
not an objective metric measured against non-functional requirements.

Upon reflection, we learned that our requirement elicitation should have been more targeted and produced a Software Requirement Specification document that we verified and agreed to conform to.
We should have made a process for ongoing requirement elicitation during development.
In the case of a knowledge gap being identified, we would research or ask questions to fill this gap,
then produce requirements and update our SRS accordingly.
This turnaround should not take more than a week to avoid blocking or reverting changes in the implementation.
The initial requirement elicitation phase may warrant a bit more time allocated.
It takes time to develop this process and learn about the problem domain.
It should not have taken us more than a month,
it is better to fail and identify knowledge gaps earlier than be as inefficient as we were.

\subsubsection{Timelines}
There were a few problems with the initial timeline from the proposal (see Appendix~\ref{ch:project-proposal}).
The first one was keeping an optional objective and excluding it from our timeline.
Our timeline represented the worst-case scenario to still deliver a prototype.
This meant our target of a high-quality project required us to outperform the timeline we proposed.

We also did not include any slack-time in our project, we included it by allocating extra time instead.
This meant components that had six weeks allocated to them were intended to be implemented in potentially a week or two in reality.
This, and keeping the optional objective implicit, made us quite lax in the early stages of the project.
It ultimately affected the timeline negatively and impacted the feasibility of the project.

The timeline was also based on two assumptions that did not hold and made it impossible to achieve.
We had agreed as a team to spend a set amount of hours regularly on the project.
This assumption was never held throughout the project.
We also based our timeline on a specific scope, architecture, and components.
The moment implementation started, we rewrote most of it because we changed our target diagram type.

\subsubsection{Well-Defined Testing Criteria}
As mentioned in our further recommendations in section~\ref{subsubsec:rec-v-v}, our testing could use improvements.
Ideally, after designing our architecture, we could have defined test selection and coverage criteria.
If we agreed on it, we could have had an automated mechanism to check our test coverage to see if we're missing tests.
This should have helped us improve our test coverage and build a better case for assurance in our tool.

This would massively improve our verification process.
It would also help with validation because it is much easier to trace well-defined test criteria.
It should also indicate to us early on when we are missing system tests for validation.
In hindsight, we realized quite late what sorts of tests we could make that are not directly linked to the functional output.
These are related to properties like portability, compatibility, and scalability, for example.